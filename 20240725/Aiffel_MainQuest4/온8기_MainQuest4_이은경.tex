\documentclass{article}

% if you need to pass options to natbib, use, e.g.:
%     \PassOptionsToPackage{numbers, compress}{natbib}
% before loading neurips_2021

% ready for submission
\usepackage{neurips_2021}

% IMPORTANT: if you are submitting attention track, please add the attention option:
% \usepackage[attention]{neurips_2021}

% to compile a preprint version, e.g., for submission to arXiv, add add the
% [preprint] option:
%     \usepackage[preprint]{neurips_2021}

% to compile a camera-ready version, add the [final] option, e.g.:
%     \usepackage[final]{neurips_2021}

% to avoid loading the natbib package, add option nonatbib:
%    \usepackage[nonatbib]{neurips_2021}

\usepackage[utf8]{inputenc} % allow utf-8 input
\usepackage[T1]{fontenc}    % use 8-bit T1 fonts
\usepackage{hyperref}       % hyperlinks
\usepackage{url}            % simple URL typesetting
\usepackage{booktabs}       % professional-quality tables
\usepackage{amsfonts}       % blackboard math symbols
\usepackage{nicefrac}       % compact symbols for 1/2, etc.
\usepackage{microtype}      % microtypography
\usepackage{xcolor}         % colors

\title{Categorization of Internet News Headlines \\ based on Bert Model}

% The \author macro works with any number of authors. There are two commands
% used to separate the names and addresses of multiple authors: \And and \AND.
%
% Using \And between authors leaves it to LaTeX to determine where to break the
% lines. Using \AND forces a line break at that point. So, if LaTeX puts 3 of 4
% authors names on the first line, and the last on the second line, try using
% \AND instead of \And before the third author name.

\author{%
  eunkyung lee.~\thanks{Use footnote for providing further information
    about author (webpage, alternative address)---\emph{not} for acknowledging
    funding agencies.} \\
  Department of Computer Science\\
  SNUT University\\
  Pittsburgh, PA 15213 \\
  \texttt{hippo@cs.cranberry-lemon.edu} \\
  % examples of more authors
  \And
  % Coauthor \\
  % Affiliation \\
  % Address \\
  \texttt{eunkyung0111@gmail.com} \\
  % \AND
  % Coauthor \\
  % Affiliation \\
  % Address \\
  % \texttt{email} \\
  % \And
  % Coauthor \\
  % Affiliation \\
  % Address \\
  % \texttt{email} \\
  % \And
  % Coauthor \\
  % Affiliation \\
   \aAddress{seoul, korea} \\
  \texttt{eunkyung0111@gmail.com} \\
}

\begin{document}

\maketitle

\begin{abstract}
  The abstract paragraph should be indented \nicefrac{1}{2}~inch (3~picas) on
  both the left- and right-hand margins. Use 10~point type, with a vertical
  spacing (leading) of 11~points.  The word \textbf{Abstract} must be centered,
  bold, and in point size 12. Two line spaces precede the abstract. The abstract
  must be limited to one paragraph.
\end{abstract}

\section{Introduction}

Introduction

\section{Background, Related works}
\label{gen_inst}

\subsection{The Need for News Categorization in the Age of Information Overload}
In modern society, internet news has become a primary source of information. Every day, millions of news articles are published online, creating an information overload that makes it difficult for users to find the information they need. To address this issue, there is a growing need for technologies that can effectively categorize news articles and make them easily accessible to users. News categorization helps by automatically classifying news articles into specific topics or categories, enabling users to quickly find the information they seek.

\subsection{Advances in Machine Learning and Natural Language Processing}
Recent advancements in machine learning and natural language processing (NLP) have significantly improved the efficiency and accuracy of news categorization tasks. Particularly, deep learning models have shown exceptional performance by effectively learning from large datasets. Among these, the BERT (Bidirectional Encoder Representations from Transformers) model, a pre-trained language model based on the transformer architecture, stands out. BERT's ability to understand context bidirectionally makes it highly effective for text classification tasks.

\subsection{The Superiority of the BERT Model}
The BERT model has set new benchmarks across various NLP tasks and has demonstrated its potential in news categorization. By leveraging bidirectional contextual information, BERT can more accurately grasp the meaning of words compared to unidirectional models. This advantage is especially beneficial when classifying complex texts like news articles, where understanding the context is crucial.

\subsection{Significance and Objectives of the Research}
This study aims to explore the use of the BERT model for categorizing internet news headlines. By doing so, it seeks to enhance the efficiency of automated news categorization systems, making it easier for users to find relevant information. News headlines, being composed of short texts, require a deep understanding of context, which makes BERT an appropriate choice for this task.

Based on this background, this study aims to verify the feasibility and efficiency of using the BERT model for categorizing internet news headlines. Through this research, we hope to contribute to the development of more advanced news categorization systems.

\section{Method}

\label{headings}
News Categorization Using BERT

Two-way Encoder Representation from Transformers (BERT)** is an innovative model in Natural Language Processing (NLP), demonstrating its excellent performance in understanding the two-way context of text.

\subsection{Data Collection and Preprocessing}

The first step of the research involves collecting a dataset of news articles. News headlines and bodies are gathered from various sources such as CNN, BBC, and Reuters, categorized by topic. The collected data undergoes the following preprocessing steps:

Deduplication: Remove duplicate articles to maintain data purity.
Normalization: Convert all text to lowercase and remove punctuation, numbers, and unnecessary whitespace.
Tokenization: Tokenize the text to match the input format required by the BERT model. BERT uses the WordPiece tokenizer.
Padding and Truncation: Add padding and truncate the input sequences to ensure a consistent length.

\subsection{Dataset Splitting}

The preprocessed dataset is divided into training, validation, and test sets. Typically, 70\% of the data is used for training, 15\% for validation, and 15\% for testing. This division helps evaluate the model's performance and prevent overfitting.


\subsection{BERT Model Setup and Fine-Tuning}
Using a pre-trained version of the BERT model, fine-tune it for the task of news categorization. The key steps are as follows:

Model Selection: Choose a version of BERT (e.g., BERT-base, BERT-large).
Input Formatting: Convert each news headline into the input format required by the BERT model.
Adding a Classification Layer: Add a fully connected layer on top of BERT's output to classify each news article into a category.
Setting Loss Function and Optimizer: Use cross-entropy loss as the loss function and AdamW as the optimizer to fine-tune the model.

\subsubsection{Model Training}
Train the BERT model using the training dataset, and monitor its performance using the validation dataset. Apply early stopping to prevent overfitting. Key training parameters include:

Number of Epochs: Set how many times the model will iterate over the training dataset.
Batch Size: Set the amount of data processed at one time.
Learning Rate: Set the learning rate used to update the model's weights.

\subsubsection{Model Evaluation}
Evaluate the trained model using the test dataset. Performance metrics include accuracy, precision, recall, and F1 score. These metrics provide an objective measure of the model's classification performance.

\subsubsection{Hyperparameter Tuning}
Optimize the model's performance by adjusting various hyperparameters such as learning rate, batch size, and number of epochs. Methods like grid search or random search can be used for hyperparameter tuning.

\subsubsection{Result Analysis and Visualization}
Analyze the final model's performance and visualize the classification results. For instance, use a confusion matrix to visually represent the classification accuracy for each category. This helps identify specific errors made by the model in particular categories and explore further improvement strategies.

\subsubsection{Model Deployment}
Finally, deploy the trained and evaluated BERT model in a real-time news categorization system. This stage involves designing a deployment strategy considering the model's response time and efficiency. The model may need to be optimized or infrastructure scaled to enhance real-time processing performance.

By following this methodology, an effective news categorization system using BERT can be implemented, helping users quickly and accurately find news articles of interest.


\section{Result}
\label{others}

These instructions apply to everyone.

\subsection{Citations within the text}

The \verb+natbib+ package will be loaded for you by default.  Citations may be
author/year or numeric, as long as you maintain internal consistency.  As to the
format of the references themselves, any style is acceptable as long as it is
used consistently.

The documentation for \verb+natbib+ may be found at
\begin{center}
  \url{http://mirrors.ctan.org/macros/latex/contrib/natbib/natnotes.pdf}
\end{center}
Of note is the command \verb+\citet+, which produces citations appropriate for
use in inline text.  For example,
\begin{verbatim}
   \citet{hasselmo} investigated\dots
\end{verbatim}
produces
\begin{quote}
  Hasselmo, et al.\ (1995) investigated\dots
\end{quote}

If you wish to load the \verb+natbib+ package with options, you may add the
following before loading the \verb+neurips_2021+ package:
\begin{verbatim}
   \PassOptionsToPackage{options}{natbib}
\end{verbatim}

If \verb+natbib+ clashes with another package you load, you can add the optional
argument \verb+nonatbib+ when loading the style file:
\begin{verbatim}
   \usepackage[nonatbib]{neurips_2021}
\end{verbatim}

As submission is double blind, refer to your own published work in the third
person. That is, use ``In the previous work of Jones et al.\ [4],'' not ``In our
previous work [4].'' If you cite your other papers that are not widely available
(e.g., a journal paper under review), use anonymous author names in the
citation, e.g., an author of the form ``A.\ Anonymous.''

\subsection{Footnotes}

Footnotes should be used sparingly.  If you do require a footnote, indicate
footnotes with a number\footnote{Sample of the first footnote.} in the
text. Place the footnotes at the bottom of the page on which they appear.
Precede the footnote with a horizontal rule of 2~inches (12~picas).

Note that footnotes are properly typeset \emph{after} punctuation
marks.\footnote{As in this example.}

\subsection{Figures}

\begin{figure}
  \centering
  \fbox{\rule[-.5cm]{0cm}{4cm} \rule[-.5cm]{4cm}{0cm}}
  \caption{Sample figure caption.}
\end{figure}

All artwork must be neat, clean, and legible. Lines should be dark enough for
purposes of reproduction. The figure number and caption always appear after the
figure. Place one line space before the figure caption and one line space after
the figure. The figure caption should be lower case (except for first word and
proper nouns); figures are numbered consecutively.

You may use color figures.  However, it is best for the figure captions and the
paper body to be legible if the paper is printed in either black/white or in
color.

\subsection{Tables}

All tables must be centered, neat, clean and legible.  The table number and
title always appear before the table.  See Table~\ref{sample-table}.

Place one line space before the table title, one line space after the
table title, and one line space after the table. The table title must
be lower case (except for first word and proper nouns); tables are
numbered consecutively.

Note that publication-quality tables \emph{do not contain vertical rules.} We
strongly suggest the use of the \verb+booktabs+ package, which allows for
typesetting high-quality, professional tables:
\begin{center}
  \url{https://www.ctan.org/pkg/booktabs}
\end{center}
This package was used to typeset Table~\ref{sample-table}.

\begin{table}
  \caption{Sample table title}
  \label{sample-table}
  \centering
  \begin{tabular}{lll}
    \toprule
    \multicolumn{2}{c}{Part}                   \\
    \cmidrule(r){1-2}
    Name     & Description     & Size ($\mu$m) \\
    \midrule
    Dendrite & Input terminal  & $\sim$100     \\
    Axon     & Output terminal & $\sim$10      \\
    Soma     & Cell body       & up to $10^6$  \\
    \bottomrule
  \end{tabular}
\end{table}

\section{Discussion, Conclusion}

Do not change any aspects of the formatting parameters in the style files.  In
particular, do not modify the width or length of the rectangle the text should
fit into, and do not change font sizes (except perhaps in the
\textbf{References} section; see below). Please note that pages should be
numbered.

\section{Acknowledgment}

Please prepare submission files with paper size ``US Letter,'' and not, for
example, ``A4.''

Fonts were the main cause of problems in the past years. Your PDF file must only
contain Type 1 or Embedded TrueType fonts. Here are a few instructions to
achieve this.

\begin{itemize}

\item You should directly generate PDF files using \verb+pdflatex+.

\item You can check which fonts a PDF files uses.  In Acrobat Reader, select the
  menu Files$>$Document Properties$>$Fonts and select Show All Fonts. You can
  also use the program \verb+pdffonts+ which comes with \verb+xpdf+ and is
  available out-of-the-box on most Linux machines.

\item The IEEE has recommendations for generating PDF files whose fonts are also
  acceptable for NeurIPS. Please see
  \url{http://www.emfield.org/icuwb2010/downloads/IEEE-PDF-SpecV32.pdf}

\item \verb+xfig+ "patterned" shapes are implemented with bitmap fonts.  Use
  "solid" shapes instead.

\item The \verb+\bbold+ package almost always uses bitmap fonts.  You should use
  the equivalent AMS Fonts:
\begin{verbatim}
   \usepackage{amsfonts}
\end{verbatim}
followed by, e.g., \verb+\mathbb{R}+, \verb+\mathbb{N}+, or \verb+\mathbb{C}+
for $\mathbb{R}$, $\mathbb{N}$ or $\mathbb{C}$.  You can also use the following
workaround for reals, natural and complex:
\begin{verbatim}
   \newcommand{\RR}{I\!\!R} %real numbers
   \newcommand{\Nat}{I\!\!N} %natural numbers
   \newcommand{\CC}{I\!\!\!\!C} %complex numbers
\end{verbatim}
Note that \verb+amsfonts+ is automatically loaded by the \verb+amssymb+ package.

\end{itemize}

If your file contains type 3 fonts or non embedded TrueType fonts, we will ask
you to fix it.

\subsection{Margins in \LaTeX{}}

Most of the margin problems come from figures positioned by hand using
\verb+\special+ or other commands. We suggest using the command
\verb+\includegraphics+ from the \verb+graphicx+ package. Always specify the
figure width as a multiple of the line width as in the example below:
\begin{verbatim}
   \usepackage[pdftex]{graphicx} ...
   \includegraphics[width=0.8\linewidth]{myfile.pdf}
\end{verbatim}
See Section 4.4 in the graphics bundle documentation
(\url{http://mirrors.ctan.org/macros/latex/required/graphics/grfguide.pdf})

A number of width problems arise when \LaTeX{} cannot properly hyphenate a
line. Please give LaTeX hyphenation hints using the \verb+\-+ command when
necessary.

\begin{ack}
Use unnumbered first level headings for the acknowledgments. All acknowledgments
go at the end of the paper before the list of references. Moreover, you are required to declare
funding (financial activities supporting the submitted work) and competing interests (related financial activities outside the submitted work).
More information about this disclosure can be found at: \url{https://neurips.cc/Conferences/2021/PaperInformation/FundingDisclosure}.

Do {\bf not} include this section in the anonymized submission, only in the final paper. You can use the \texttt{ack} environment provided in the style file to autmoatically hide this section in the anonymized submission.
\end{ack}

\section*{References}

References follow the acknowledgments. Use unnumbered first-level heading for
the references. Any choice of citation style is acceptable as long as you are
consistent. It is permissible to reduce the font size to \verb+small+ (9 point)
when listing the references.
Note that the Reference section does not count towards the page limit.
\medskip

{
\small

[1] Alexander, J.A.\ \& Mozer, M.C.\ (1995) Template-based algorithms for
connectionist rule extraction. In G.\ Tesauro, D.S.\ Touretzky and T.K.\ Leen
(eds.), {\it Advances in Neural Information Processing Systems 7},
pp.\ 609--616. Cambridge, MA: MIT Press.

[2] Bower, J.M.\ \& Beeman, D.\ (1995) {\it The Book of GENESIS: Exploring
  Realistic Neural Models with the GEneral NEural SImulation System.}  New York:
TELOS/Springer--Verlag.

[3] Hasselmo, M.E., Schnell, E.\ \& Barkai, E.\ (1995) Dynamics of learning and
recall at excitatory recurrent synapses and cholinergic modulation in rat
hippocampal region CA3. {\it Journal of Neuroscience} {\bf 15}(7):5249-5262.
}

%%%%%%%%%%%%%%%%%%%%%%%%%%%%%%%%%%%%%%%%%%%%%%%%%%%%%%%%%%%%

\end{document}